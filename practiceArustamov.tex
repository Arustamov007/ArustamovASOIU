%% -*- coding: utf-8 -*-
\documentclass[12pt,a4paper]{scrartcl} 
\usepackage[utf8]{inputenc}
\usepackage[english,russian]{babel}
\usepackage{indentfirst}
\usepackage{misccorr}
\usepackage{graphicx}
\usepackage{amsmath}
\begin{document}
	\begin{titlepage}
		\begin{center}
			\large
			МИНИСТЕРСТВО НАУКИ И ВЫСШЕГО ОБРАЗОВАНИЯ РОССИЙСКОЙ ФЕДЕРАЦИИ
			
			Федеральное государственное бюджетное образовательное учреждение высшего образования
			
			\textbf{АДЫГЕЙСКИЙ ГОСУДАРСТВЕННЫЙ УНИВЕРСИТЕТ}
			\vspace{0.25cm}
			
			Инженерно-физический факультет
			
			Кафедра автоматизированных систем обработки информации и управления
			\vfill

			\vfill
			
			\textsc{Отчет по практике}\\[5mm]
			
			\LARGE\textit{Вариант 7}
			
			{\LARGE Нахождение определителя матрицы}
			\bigskip
			
			1 курс, группа 1ИВТ АСОИУ
		\end{center}
		\vfill
		
		\newlength{\ML}
		\settowidth{\ML}{«\underline{\hspace{0.7cm}}» \underline{\hspace{2cm}}}
		\hfill\begin{minipage}{0.5\textwidth}
			Выполнил:\\
			\underline{\hspace{\ML}} Г.\,К.~Арустамов\\
			«\underline{\hspace{0.7cm}}» \underline{\hspace{2cm}} 2024 г.
		\end{minipage}%
		\bigskip
		
		\hfill\begin{minipage}{0.5\textwidth}
			Руководитель:\\
			\underline{\hspace{\ML}} С.\,В.~Теплоухов\\
			«\underline{\hspace{0.7cm}}» \underline{\hspace{2cm}} 2024 г.
		\end{minipage}%
		
		
		\vfill
		
		
		
		\begin{center}
			
			Майкоп, 2024 г.
		\end{center}
	\end{titlepage}
\LARGE{Содержание}

\begin{enumerate}
	\item Задача
	\item Пример кода, решающего данную задачу
	\item Скриншот работы программы
\end{enumerate}
\section{Задача}
Найти определитель матрицы.
\section{Пример кода}
\label{sec:exp:code}
\begin{verbatim}
#include <iostream>
#include <cmath>

using namespace std;

//Данная программа вычисляет определитель
 матрицы используя рекурсию 


void clearMemory(int** a, int n) { //Функция 
	освобождения памяти, выделенной под 
	двумерный динамический массив
	for (int i = 0; i < n; i++) {
		delete[] a[i];
	}
	delete[] a;
}

int findDet(int** a, int n) { //Рекурсивная 
	функция вычисления определителя матрицы
	if (n == 1)
	return a[0][0];
	else if (n == 2)
	return a[0][0] * a[1][1] - a[0][1] * a[1][0];
	else {
		int d = 0;
		for (int k = 0; k < n; k++) {
			int** m = new int* [n - 1];
			for (int i = 0; i < n - 1; i++) {
				m[i] = new int[n - 1];
			}
			for (int i = 1; i < n; i++) {
				int t = 0;
				for (int j = 0; j < n; j++) {
					if (j == k)
					continue;
					m[i - 1][t] = a[i][j];
					t++;
				}
				
			}
			d += pow(-1, k + 2) * a[0][k] * 
			findDet(m, n - 1);
			clearMemory(m, n - 1); //Освобождаем память,
			 выделенную под алгебраическое дополнение
		}
		
		return d; //Возвращаем определитель матрицы
	}
}

int main() {
	int n; //Объявляем целочисленную переменную
	cout << "Enter a matrix size:\n";
	cout << "n = ";
	
	cin >> n; //Вводим размерность матрицы
	int** a = new int* [n]; //Объявляем двумерный
	 целочисленный динамический массив (матрицу)
	for (int i = 0; i < n; i++) {
		a[i] = new int[n];
		
	}
	cout << "Enter a matrix:\n";
	for (int i = 0; i < n; i++) {
		for (int j = 0; j < n; j++) {
			cin >> a[i][j]; //Вводим элементы матрицы
		}
		
	}
	cout << "Found determinant: " << findDet(a, n)
	 << "\n"; //Вызываем рекурсивную функцию 
	 вычисления определителя матрицы
	 
	clearMemory(a, n); //Освобождаем память,
	 выделенную под исходную матрицу
	 
	system("pause");
	return 0;
	

}
	
	
	
\end{verbatim}
\vfill




\section{Скриншот работы программы}
\label{sec:picexample}
\begin{figure}[h]
	\centering
	\includegraphics[width=0.9\textwidth]{practic.png}
	\caption{Результат}\label{fig:par}
\end{figure}
\end{document}